%%%%%%%%%%%%%%
%% Run LaTeX on this file several times to get Table of Contents,
%% cross-references, and citations.

%% If you have font problems, you may edit the w-bookps.sty file
%% to customize the font names to match those on your system.

%% w-bksamp.tex. Current Version: Feb 16, 2012
%%%%%%%%%%%%%%%%%%%%%%%%%%%%%%%%%%%%%%%%%%%%%%%%%%%%%%%%%%%%%%%%
%
%  Sample file for
%  Wiley Book Style, Design No.: SD 001B, 7x10
%  Wiley Book Style, Design No.: SD 004B, 6x9
%
%
%  Prepared by Amy Hendrickson, TeXnology Inc.
%  http://www.texnology.com
%%%%%%%%%%%%%%%%%%%%%%%%%%%%%%%%%%%%%%%%%%%%%%%%%%%%%%%%%%%%%%%%

%%%%%%%%%%%%%
% 7x10
%\documentclass{wileySev}

% 6x9
\documentclass{wileySix}

\usepackage{graphicx}

%%%%%%%
%% for times math: However, this package disables bold math (!)
%% \mathbf{x} will still work, but you will not have bold math
%% in section heads or chapter titles. If you don't use math
%% in those environments, mathptmx might be a good choice.

% \usepackage{mathptmx}

% For PostScript text
\usepackage{w-bookps}

%%%%%%%%%%%%%%%%%%%%%%%%%%%%%%%%%%%%%%%%%%%%%%%%%%%%%%%%%%%%%%%%
%% Other packages you might want to use:

% for chapter bibliography made with BibTeX
% \usepackage{chapterbib}

% for multiple indices
% \usepackage{multind}

% for answers to problems
% \usepackage{answers}

%%%%%%%%%%%%%%%%%%%%%%%%%%%%%%
%% Change options here if you want:
%%
%% How many levels of section head would you like numbered?
%% 0= no section numbers, 1= section, 2= subsection, 3= subsubsection
%%==>>
\setcounter{secnumdepth}{3}

%% How many levels of section head would you like to appear in the
%% Table of Contents?
%% 0= chapter titles, 1= section titles, 2= subsection titles, 
%% 3= subsubsection titles.
%%==>>
\setcounter{tocdepth}{2}

%% Cropmarks? good for final page makeup
%% \docropmarks

%%%%%%%%%%%%%%%%%%%%%%%%%%%%%%
%
% DRAFT
%
% Uncomment to get double spacing between lines, current date and time
% printed at bottom of page.
% \draft
% (If you want to keep tables from becoming double spaced also uncomment
% this):
% \renewcommand{\arraystretch}{0.6}
%%%%%%%%%%%%%%%%%%%%%%%%%%%%%%

%%%%%%% Demo of section head containing sample macro:
%% To get a macro to expand correctly in a section head, with upper and
%% lower case math, put the definition and set the box 
%% before \begin{document}, so that when it appears in the 
%% table of contents it will also work:

\newcommand{\VT}[1]{\ensuremath{{V_{T#1}}}}

%% use a box to expand the macro before we put it into the section head:

\newbox\sectsavebox
\setbox\sectsavebox=\hbox{\boldmath\VT{xyz}}

%%%%%%%%%%%%%%%%% End Demo


\begin{document}


\booktitle{Survey Methodology}
\subtitle{This is the Subtitle}

\authors{Robert M. Groves\\
\affil{Universitat de les Illes Balears}
Floyd J. Fowler, Jr.\\
\affil{University of New Mexico}
}

\offprintinfo{Survey Methodology, Second Edition}{Robert M. Groves}

%% Can use \\ if title, and edition are too wide, ie,
%% \offprintinfo{Survey Methodology,\\ Second Edition}{Robert M. Groves}

%%%%%%%%%%%%%%%%%%%%%%%%%%%%%%
%% 
\halftitlepage

\titlepage


\begin{copyrightpage}{2007}
Survey Methodology / Robert M. Groves . . . [et al.].
\       p. cm.---(Wiley series in survey methodology)
\    ``Wiley-Interscience."
\    Includes bibliographical references and index.
\    ISBN 0-471-48348-6 (pbk.)
\    1. Surveys---Methodology.  2. Social 
\  sciences---Research---Statistical methods.  I. Groves, Robert M.  II. %
Series.\\

HA31.2.S873 2007
001.4'33---dc22                                             2004044064
\end{copyrightpage}



\dedication{To my parents}

\begin{contributors}
\name{Masayki Abe,} Fujitsu Laboratories Ltd., Fujitsu Limited, Atsugi,
Japan

\name{L. A. Akers,} Center for Solid State Electronics Research, Arizona
State University, Tempe, Arizona

\name{G. H. Bernstein,} Department of Electrical and
Computer Engineering, University of Notre Dame, Notre Dame, South Bend, 
Indiana; formerly of
Center for Solid State Electronics Research, Arizona
State University, Tempe, Arizona 
\end{contributors}

\contentsinbrief
\tableofcontents
\listoffigures
\listoftables


\begin{foreword}
This is the foreword to the book.
\end{foreword}

\begin{preface}
This is an example preface.
This is an example preface.
This is an example preface.
This is an example preface.

\prefaceauthor{R. K. Watts}
\where{Durham, North Carolina\\
September, 2007}

\end{preface}


\begin{acknowledgments}
From Dr.~Jay Young, consultant from Silver Spring, Maryland, I received
the initial push to even consider writing this book. Jay was a constant
``peer reader'' and very welcome advisor durying this year-long process.


To all these wonderful people I owe a deep sense of gratitude especially now
that this project has been completed.
\authorinitials{G. T. S.}
\end{acknowledgments}

\begin{acronyms}
\acro{ACGIH}{American Conference of Governmental Industrial Hygienists}
\acro{AEC}{Atomic Energy Commission}
\acro{OSHA}{Occupational Health and Safety Commission}
\acro{SAMA}{Scientific Apparatus Makers Association}
\end{acronyms}

\begin{glossary}
\term{NormGibbs}Draw a sample from a posterior distribution
of data with an unknown mean and variance using Gibbs sampling.

\term{pNull}Test a one sided hypothesis from a numberically
specified posterior CDF or from a sample from the posterior

\term{sintegral}A numerical integration using Simpson's rule
\end{glossary}

\begin{symbols}
\term{A}Amplitude

\term{\hbox{\&}}Propositional logic symbol 

\term{a}Filter Coefficient

\bigskip

\term{\mathcal{B}}Number of Beats
\end{symbols}

\begin{introduction}

%% optional, but if you want to list author:

\introauthor{Catherine Clark, PhD.}
{Harvard School of Public Health\\
Boston, MA, USA}

The era of modern \index{microelectronics}\index{microelectronics!modern} 
began in 1958 with the invention of the
integrated circuit by J.~S.~Kilby
 of Texas Instruments \cite{kilby}.
His first chip is shown in Fig.~I. For comparison,
Fig.~I.2 shows a modern microprocessor chip, \cite{beren}.


This is the introduction.
This is the introduction.
This is the introduction.
This is the introduction.
This is the introduction.
This is the introduction.

\begin{equation}
ABC {\cal DEF} \alpha\beta\Gamma\Delta\sum^{abc}_{def}
\end{equation}


\begin{chapreferences}{3.}
\bibitem{zkilby}J. S. Kilby,
``Invention of the Integrated Circuit,'' {\it IEEE Trans. Electron Devices,}
{\bf ED-23,} 648 (1976).

\bibitem{zhamming}R. W. Hamming,
                 {\it Numerical Methods for Scientists and 
                 Engineers}, Chapter N-1, McGraw-Hill, 
                 New York, 1962.

\bibitem{zHu}J. Lee, K. Mayaram, and C. Hu, ``A Theoretical
               Study of Gate/Drain Offset in LDD MOSFETs''
                     {\it IEEE Electron Device Lett.,} {\bf EDL-7}(3). 152 
                     (1986).
\end{chapreferences}
\end{introduction}

\chapter{Installation}

\chapter{Your First Document}

\chapter{Structuring Your Document (Section and Paragraph)}

\chapter{Packages Explained}

\chapter{Typesetting Math in Latex}

\chapter{Adding a Picture}

\chapter{Generate a Table of Contents}

\chapter{Adding Bibliography}

\chapter{Adding Footnotes}

\chapter{Create Tables with Latex}

\chapter{Using Tables the Smart Way}

\chapter{Plots Visualizing Your Data With Pgfgplots}

\chapter{Electric Circuit With Circuitikz}

\chapter{Source Code Hightlighting in Latex using the Listing Package (Listing)}

\begin{references}{3.}
\bibitem{kilby}J. S. Kilby,
``Invention of the Integrated Circuit,'' {\it IEEE Trans. Electron Devices,}
{\bf ED-23,} 648 (1976).

\bibitem{hamming}R. W. Hamming,
                 {\it Numerical Methods for Scientists and 
                 Engineers}, Chapter N-1, McGraw-Hill, 
                 New York, 1962.

\bibitem{Hu}J. Lee, K. Mayaram, and C. Hu, ``A Theoretical
               Study of Gate/Drain Offset in LDD MOSFETs''
                     {\it IEEE Electron Device Lett.,} {\bf EDL-7}(3). 152 
                     (1986).

\bibitem{beren}A. Berenbaum, 
B. W. Colbry, D.R. Ditzel, R. D Freeman, and 
K.J. O'Connor, ``A Pipelined 32b Microprocessor with 13 kb of Cache Memory,''
{it Int. Solid State Circuit Conf., Dig. Tech. Pap.,} p. 34 (1987).
\end{references}


\begin{references}{Ham62}
\bibitem[Kil76]{kilb}J. S. Kilby,
``Invention of the Integrated Circuit,'' {\it IEEE Trans. Electron Devices,}
{\bf ED-23,} 648 (1976).

\bibitem[Ham62]{hamm}R. W. Hamming,
                 {\it Numerical Methods for Scientists and 
                 Engineers}, Chapter N-1, McGraw-Hill, 
                 New York, 1962.

\bibitem[Hu86]{lee}J. Lee, K. Mayaram, and C. Hu, ``A Theoretical
               Study of Gate/Drain Offset in LDD MOSFETs''
                     {\it IEEE Electron Device Lett.,} {\bf EDL-7}(3). 152 
                     (1986).

\bibitem[Ber87]{berm}A. Berenbaum, 
B. W. Colbry, D.R. Ditzel, R. D Freeman, and 
K.J. O'Connor, ``A Pipelined 32b Microprocessor with 13 kb of Cache Memory,''
{it Int. Solid State Circuit Conf., Dig. Tech. Pap.,} p. 34 (1987).

\end{references}



%%%%%%%%%%%%%%%
%%  The default LaTeX Index
%%  Don't need to add any commands before \begin{document}
\printindex

%%%% Making an index
%% 
%% 1. Make index entries, don't leave any spaces so that they
%% will be sorted correctly.
%% 
%% \index{term}
%% \index{term!subterm}
%% \index{term!subterm!subsubterm}
%% 
%% 2. Run LaTeX several times to produce <filename>.idx
%% 
%% 3. On command line, type  makeindx <filename> which
%% will produce <filename>.ind 
%% 
%% 4. Type \printindex to make the index appear in your book.
%% 
%% 5. If you would like to edit <filename>.ind 
%% you may do so. See docs.pdf for more information.
%% 
%%%%%%%%%%%%%%%%%%%%%%%%%%%%%%

%%%%%%%%%%%%%% Making Multiple Indices %%%%%%%%%%%%%%%%
%% 1. 
%% \usepackage{multind}
%% \makeindex{book}
%% \makeindex{authors}
%% \begin{document}
%% 
%% 2.
%% % add index terms to your book, ie,
%% \index{book}{A term to go to the topic index}
%% \index{authors}{Put this author in the author index}
%% 
%% \index{book}{Cows}
%% \index{book}{Cows!Jersey}
%% \index{book}{Cows!Jersey!Brown}
%% 
%% \index{author}{Douglas Adams}
%% \index{author}{Boethius}
%% \index{author}{Mark Twain}
%% 
%% 3. On command line type 
%% makeindex topic 
%% makeindex authors
%% 
%% 4.
%% this is a Wiley command to make the indices print:
%% \multiprintindex{book}{Topic index}
%% \multiprintindex{authors}{Author index}

\end{document}

